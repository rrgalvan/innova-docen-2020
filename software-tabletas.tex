\documentclass[11pt]{article}

\usepackage[utf8]{inputenc}
\usepackage[spanish]{babel}
\usepackage[margin={3cm,2cm}]{geometry}
\usepackage{hyperref}

\title{Software para tabletas y pantallas gráficas}
\date{\today}

\newcommand{\nodoc}{No documentado}

\begin{document}
\maketitle

\section{Introducción}
Entre los objetivos del proyecto de innovación docente «Docencia en
entornos no presenciales: tecnología sustitutiva de la pizarra y
refuerzos en aula invertida» se recogieron las siguientes tareas:

\begin{itemize}
\item Investigar el software disponible para utilizar
  \textbf{tabletas} a modo de pizarra digital y su aplicabilidad en la
  docencia de tipo asíncrono o no presencial, con especial énfasis en
  software libre o gratuito. Realizar un informe con las aplicaciones
  estudiadas.

\item Investigar el software disponible para utilizar una \textbf{pantalla
  gráfica} a modo de pizarra digital y su aplicabilidad en la docencia
  de tipo asíncrono o no presencial. Se pondrá poniendo especial
  énfasis en software libre. Realizar un informe con las aplicaciones
  estudiadas.
\end{itemize}

El presente documento recoge los correspondientes informes sobre las aplicaciones estudiadas.

\section{Software para tabletas}

Software disponible en forma de aplicaciones (\textit{apps}) para
entornos Android o iOS (Mac). Con frecuencia, también existen
versiones que posibilitan su uso en entornos web.

\subsection{Xodo}

\begin{itemize}
\item Sitio web: \url{https://www.xodo.com}
\item Descripción:
\item Plataformas:
  \begin{itemize}
  \item Dispositivos Android: \nodoc
  \item Dispositivos iOS: \nodoc
  \item Entornos Web: \nodoc
  \end{itemize}
\item Licencia: \nodoc
\end{itemize}

\subsection{Foxit}

\begin{itemize}
\item Sitio web: \url{https://www.foxitsoftware.com}
\item Descripción:
\item Plataformas para las que está disponible:
  \begin{itemize}
  \item Dispositivos Android: \nodoc
  \item Dispositivos iOS: \nodoc
  \item Entornos Web: \nodoc
  \end{itemize}
\item Licencia: \nodoc
\end{itemize}

\section{Software para pantallas gráficas}

Software que se ejecuta en un ordenador, usando un sistema operativo
determinado, y mediante pantallas gráficas permite realizar dibujos y
anotaciones.

\subsection{Foxit}

\begin{itemize}
\item Sitio web: \url{https://www.foxitsoftware.com}
\item Descripción:
\item Plataformas para las que está disponible:
  \begin{itemize}
  \item Sistema operativo Windows: \nodoc
  \item Sistema operativo MacOS: \nodoc
  \item Sistema operativo GNU/Linux: \nodoc
  \item Entornos Web: \nodoc
  \end{itemize}
\item Licencia: \nodoc
\end{itemize}


\subsection{Xournal++}

\begin{itemize}
\item Sitio web: \url{https://github.com/xournalpp/xournalpp}
\item Descripción:
\item Plataformas para las que está disponible:
  \begin{itemize}
  \item Sistema operativo Windows: \nodoc
  \item Sistema operativo MacOS: \nodoc
  \item Sistema operativo GNU/Linux: \nodoc
  \item Entornos Web: \nodoc
  \end{itemize}
\item Licencia: \nodoc
\end{itemize}

\end{document}
