\documentclass[11pt]{article}

\usepackage[utf8]{inputenc}
\usepackage[spanish]{babel}
\usepackage[margin={3cm,2cm}]{geometry}
\usepackage{hyperref}
\usepackage[dvipsnames]{xcolor}
\usepackage{eurosym}

\title{Software para tabletas y pantallas gráficas}
\date{\today}

\newcommand{\nodoc}{No documentado}
\newcommand{\libre}{\colorbox{YellowGreen}{Libre}}
\newcommand{\gratis}{\colorbox{SeaGreen}{Gratuita}}
\newcommand{\fw}{\colorbox{YellowGreen}{FreeWare}}
\newcommand{\pago}{\colorbox{Red}{De Pago}}

\begin{document}
\maketitle

\section{Introducción}
Entre los objetivos del proyecto de innovación docente «Docencia en
entornos no presenciales: tecnología sustitutiva de la pizarra y
refuerzos en aula invertida» se recogieron las siguientes tareas:

\begin{itemize}
\item Investigar el software disponible para utilizar
  \textbf{tabletas} a modo de pizarra digital y su aplicabilidad en la
  docencia de tipo asíncrono o no presencial, con especial énfasis en
  software libre o gratuito. Realizar un informe con las aplicaciones
  estudiadas.

\item Investigar el software disponible para utilizar una \textbf{pantalla
  gráfica} a modo de pizarra digital y su aplicabilidad en la docencia
  de tipo asíncrono o no presencial. Se pondrá poniendo especial
  énfasis en software libre. Realizar un informe con las aplicaciones
  estudiadas.
\end{itemize}

El presente documento recoge los correspondientes informes sobre las aplicaciones estudiadas.

\section{Software para tabletas}

Software disponible en forma de aplicaciones (\textit{apps}) para
entornos Android o iOS (Mac) que permiten realizar anotaciones
mediante el lápiz de una tableta. Muchas de ellas permiten anotar de
esta forma ficheros PDF. Con frecuencia, también existen versiones que
posibilitan su uso en entornos web.

\subsection{Xodo}

\begin{itemize}
\item Sitio web: \url{https://www.xodo.com}
\item Descripción:
Software lector de PDF con la posibilidad de añadir manualmente notas
con el lápiz de la tableta. Disponible en forma de aplicaciones
(\textit{apps}) para entornos Android o iOS (Mac).
\item Plataformas:
  \begin{itemize}
  \item Dispositivos Android: Sí
  \item Dispositivos iOS: Sí
  \item Entornos Web: Sí (con limitaciones)
  \end{itemize}
\item Licencia: \gratis
\end{itemize}

\subsection{Foxit}


\begin{itemize}
\item Sitio web: \url{https://www.foxitsoftware.com}
\item Descripción:
Lector de PDF que permite añadir notas escritas con
el lápiz de la tableta.
\item Plataformas para las que está disponible:
  \begin{itemize}
  \item Dispositivos Android: Sí
  \item Dispositivos iOS: Sí
  \item Entornos Web: Sí (previo registro)
  \end{itemize}
\item Licencia: \gratis
\end{itemize}

\subsection{OneNote}

Aplicación para la escritura de notas y diagramas con el lápiz de la tableta. Aunque sea posible la anotación de PDF, en principio esta aplicación no está orientada a ello.

\begin{itemize}
\item Sitio web: \url{http://www.onenote.com}
\item Descripción: Aplicación de notas con un lápiz/dedo sobre la pantalla o con un teclado. Interfaz intuitiva y minimalista. Interesante su opción de ``colaboración multiusuario'', que permite que varias personas puedan ver/editar una misma nota al mismo tiempo. Aunque sea posible la anotación de PDF, en principio esta aplicación no está orientada a ello.

\item Plataformas para las que está disponible:
  \begin{itemize}
  \item Dispositivos Android: Si
  \item Dispositivos iOS: Si
  \item Entornos Web: Si
  \end{itemize}
\item Licencia: \gratis
\end{itemize}

\subsection{Evernote}

\begin{itemize}
\item Sitio web: \url{https://evernote.com}
\item Descripción: Aplicación par la organización de información personal mediante el archivo de notas. Éstas pueden realizarse manualmente, con el lápiz de una tableta o pantalla idgitalizadora.
\item Plataformas:
  \begin{itemize}
  \item Dispositivos Android: Sí
  \item Dispositivos iOS: Sí
  \item Entornos Web: Sí
  \end{itemize}
\item Licencia: \gratis, \textbf{previo registro}, también versión \pago
\end{itemize}

\subsection{Notability}

\begin{itemize}
\item Sitio web: \url{https://www.gingerlabs.com}
\item Descripción: Siendo una de las aplicaciones de pago mejor valorada por los usuarios de la App Store, Notability es un editor de ``Cuadernos Digitales'' el cuál te permite tomar notas o apuntes sobre los distintos cuadernos que desees crear dentro de la aplicación.
\\
Uno de sus puntos más fuertes es la ayuda a la escritura y creación de figuras y la capacidad de personalización que ofrece para tomar las notas al completo gusto del usuario.
\\
Si posees la aplicación en distintos productos de Apple se guardan y sincronizan automáticamente todos los cuadernos en la nube pudiendo editar o tomar apuntes en ellos sobre cualquiera de los dispositivos en cualquier momento.
\\
Tiene la capacidad de poder insertar en los cuadernos archivos con distintas extensiones (.pdf, .png, .text,....) e incluso modificarlos.
\\
Desde la propia aplicación no se puede compartir la nota para que los demás usuarios puedan verla en tiempo real, pero sí tiene un modo presentación si se detecta que se está compartiendo pantalla, ofrece herramientas como un puntero para señalar en la nota.
\item Plataformas para las que está disponible:
  \begin{itemize}
  \item Dispositivos Android: No
  \item Dispositivos iOS: Si
  \item Entornos Web: Sólo macOS
  \end{itemize}
\item Licencia: \pago  \ 9.99\euro
\end{itemize}

\newpage

\section{Software para pantallas gráficas}

Software que se ejecuta en un ordenador, usando un sistema operativo
determinado, y mediante pantallas gráficas permite realizar dibujos y
anotaciones.

\subsection{Foxit}

\begin{itemize}
\item Sitio web: \url{https://www.foxitsoftware.com}
\item Descripción:
\item Plataformas para las que está disponible:
  \begin{itemize}
  \item Sistema operativo Windows: \nodoc
  \item Sistema operativo MacOS: \nodoc
  \item Sistema operativo GNU/Linux: \nodoc
  \item Entornos Web: \nodoc
  \end{itemize}
\item Licencia: \gratis
\end{itemize}


\subsection{Xournal++}

\begin{itemize}
\item Sitio web: \url{https://github.com/xournalpp/xournalpp}
\item Descripción:
\item Plataformas para las que está disponible:
  \begin{itemize}
  \item Sistema operativo Windows: \nodoc
  \item Sistema operativo MacOS: \nodoc
  \item Sistema operativo GNU/Linux: \nodoc
  \item Entornos Web: \nodoc
  \end{itemize}
\item Licencia: \libre
\end{itemize}


\subsection{Google Jamboard}

\begin{itemize}
\item Sitio web: \url{https://edu.google.com/intl/es-419/products/jamboard/}
\item Descripción: Es una pizarra digital basada en la nube. Cuenta con 20 hojas de pizarra y es fácil y rápida de manejar.
\item Plataformas para las que está disponible:
  \begin{itemize}
  \item Sistema operativo Windows: \nodoc
  \item Sistema operativo MacOS: \nodoc
  \item Entornos Web: \nodoc
  \end{itemize}
\item Licencia: \gratis
\end{itemize}

\subsection{OpenBoard}

\begin{itemize}
\item Sitio web: \url{https://openboard.ch/download.en.html}
\item Descripción: OpenBoard es un software de código abierto ( GPLv3 License ) diseñado principalmente para su uso en escuelas y univerisades. Es fácil de manejar, tiene una rápida respuesta y se puede utilizar en pizarras gráficas o con una configuración de doble pantalla.

\item Plataformas para las que está disponible:
  \begin{itemize}
  \item Sistema operativo Windows: \nodoc
  \item Sistema operativo MacOS: \nodoc
  \item Sistema operativo GNU/Linux: \nodoc
  \item Entornos Web: \nodoc
  \end{itemize}
\item Licencia: \libre
\end{itemize}


\subsection{Groupboard}

\begin{itemize}
\item Sitio web: \url{https://www.groupboard.com/products/}
\item Descripción: Es una pizarra colaborativa de forma online, que tiene un servicio básico de forma gratuita y un servicio más avanzado de forma no gratuita.
\item Plataformas para las que está disponible:
  \begin{itemize}
  \item Sistema operativo Windows: \nodoc
  \item Sistema operativo MacOS: \nodoc
  \item Sistema operativo GNU/Linux: \nodoc
  \item Entornos Web: \nodoc
  \end{itemize}
\item Licencia: \gratis Versión avanzada de pago.
\end{itemize}

\subsection{Notebookcast}

\begin{itemize}
\item Sitio web: \url{https://www.notebookcast.com/}
\item Descripción: Notebookcast es una pizarra online compartida multi usuario, para trabajar en tiempo real.
\item Plataformas para las que está disponible:
  \begin{itemize}
  \item Sistema operativo Windows: \nodoc
  \item Sistema operativo MacOS: \nodoc
  \item Entornos Web: \nodoc
  \end{itemize}
\item Licencia: \gratis
\end{itemize}



\end{document}
